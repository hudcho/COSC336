\iffalse

INSTRUCTIONS: (if this is not lecture1.tex, use the right file name)

  Clip out the ********* INSERT HERE ********* bits below and insert
appropriate TeX code.  Once you are done with your file, run

  ``latex lecture1.tex''

from a UNIX prompt.  If your LaTeX code is clean, the latex will exit
back to a prompt.  Once this is done, run

  ``dvips lecture1.dvi''

which should print your file to the nearest printer.  There will be
residual files called lecture1.log, lecture1.aux, and lecture1.dvi.
All these can be deleted, but do not delete lecture1.tex.
\fi
%
\documentclass[11pt]{article}
\usepackage{amsfonts}
\usepackage{amsmath}
\usepackage{latexsym}
\usepackage{hyperref}
\usepackage{listings}


\lstdefinestyle{mystyle}{
  basicstyle=\ttfamily\footnotesize,
  breakatwhitespace=false,         
  breaklines=true,                 
  captionpos=b,                    
  keepspaces=true,                 
  numbers=left,                    
  numbersep=5pt,                  
  showspaces=false,                
  showstringspaces=false,
  showtabs=false,                  
  tabsize=2
}
\lstset{style=mystyle}

\hypersetup{
    colorlinks=true,
    linkcolor=blue,
    filecolor=magenta,      
    urlcolor=cyan,
}
 
\urlstyle{same}

\setlength{\oddsidemargin}{.25in}
\setlength{\evensidemargin}{.25in}
\setlength{\textwidth}{6in}
\setlength{\topmargin}{-0.4in}
\setlength{\textheight}{8.5in}

\newcommand{\handout}[5]{
   %\renewcommand{\thepage}{#1-\arabic{page}}
   \noindent
   \begin{center}
   \framebox{
      \vbox{
    \hbox to 5.78in { {\bf Data Structures and Algorithms} \hfill #2 }
       \vspace{4mm}
       \hbox to 5.78in { {\Large \hfill #5  \hfill} }
       \vspace{2mm}
       \hbox to 5.78in { {\it #3 \hfill #4} }
      }
   }
   \end{center}
   \vspace*{4mm}
}

\newcommand{\lecture}[3]{\handout{L#1}{#2}{}{}{#1}}

\def\squarebox#1{\hbox to #1{\hfill\vbox to #1{\vfill}}}
\def\qed{\hspace*{\fill}
        \vbox{\hrule\hbox{\vrule\squarebox{.667em}\vrule}\hrule}}
\newenvironment{solution}{\begin{trivlist}\item[]{\bf Solution:}}
                      {\qed \end{trivlist}}
\newenvironment{solsketch}{\begin{trivlist}\item[]{\bf Solution Sketch:}}
                      {\qed \end{trivlist}}
\newenvironment{proof}{\begin{trivlist}\item[]{\bf Proof:}}
                      {\qed \end{trivlist}}

\newtheorem{theorem}{Theorem}
\newtheorem{corollary}[theorem]{Corollary}
\newtheorem{lemma}[theorem]{Lemma}
\newtheorem{observation}[theorem]{Observation}
\newtheorem{remark}[theorem]{Remark}
\newtheorem{proposition}[theorem]{Proposition}
\newtheorem{definition}[theorem]{Definition}
\newtheorem{Assertion}[theorem]{Assertion}
\newtheorem{fact}[theorem]{Fact}
\newtheorem{hypothesis}[theorem]{Hypothesis}
%\newtheorem{observation}[theorem]{Observation}
%\newtheorem{proposition}[theorem]{Proposition}
\newtheorem{claim}[theorem]{Claim}
\newtheorem{assumption}[theorem]{Assumption}

%Put more macros here, as needed.
\newcommand{\al}{\alpha}
\newcommand{\Z}{\mathbb Z}
\newcommand{\jac}[2]{\left(\frac{#1}{#2}\right)}
\newcommand{\set}[1]{\{#1\}}

\def\ppt{{\sf PPT}}
\def\poly{{\sf poly}}
\def\negl{{\sf negl}}
\def\owf{{\sf OWF}}
\def\owp{{\sf OWP}}
\def\tdp{{\sf TDP}}
\def\prg{{\sf PRG}}
\def\prf{{\sf PRF}}

%end of macros
\begin{document}
\fbox{
\vbox{
\begin{flushleft}
Hudson Cho, Ryan Wilson, Jesse Washburn, Colin Shuster, Samhith Patibandla\\  % authors' names
COSC 336 \\  %class
03/4/2025\\  % date
\end{flushleft}
\center{\Large{\textbf{Assignment 3}}}
%\end{mdframed}
} % end vbox
} % end fbox
\vline


\textbf{Instructions.}
\begin{enumerate}
\item Due date and time: As indicated on Blackboard. 
\item This is a team assignment. Work in teams of 3-4 students.  Submit on Blackboard one assignment per team, with the names of all students making the team. 
\item The exercises will not be graded, but you still need to present your best attempt to solve them. If you do not know how to solve an exercise, say it.  This will give me feedback about your understanding of the theoretical concepts.
\item Your programs must be written in Java.

\item Write your programs neatly - imagine yourself grading your program and see if it is easy to read and understand. 

Comment your programs reasonably: there is no need to comment lines like "i++" but do include brief comments describing the main purpose of a specific block of lines.
\item  You will submit on \textbf{Blackboard} 3 files.  

The \textbf{1-st file} is a pdf file (produced ideally with latex and Overleaf) and it will contain the following:
\begin{enumerate}
\item The solution to the Exercises (see the remark above).
\item   A short description of your algorithms for the Programming Task1, where you explain your algorithms. Focus on how you have modified MERGE (for task 1), and on the relations between subproblems for the dynamic algorithm (for task 2)..
\item    Tables clearly labeled with the results your programs give for the data sets indicated for the programming tasks. 
\item   The java code (so that the grader can make observations) of the  2 programs (for task 1 and for task 2).
\end{enumerate}


The \textbf{2-nd file} is the .java file containing the java source code for Programming Task 1.


The \textbf{3-rd file} is the .java file containing the java source code for Programming Task 2.
\end{enumerate}
\newpage

\textbf{Exercise 1.}  Analyze the following recurrences using the method that is indicated. In case you use the Master Theorem, state what the corresponding values of $a$, $b$, and $f(n)$ are and how
you determined which case of the theorem applies. 

\begin{itemize}
\item  $T(n) = 3 T\left(\frac{n}{4}\right) + 3$. Use the Master Theorem to find a $\Theta()$ evaluation, or say "Master Theorem cannot be used", if this is the case.
\item  $T(n) = 2 T\left(\frac{n}{2}\right) + 3n$. Use the Master Theorem to find a $\Theta()$ evaluation, or say "Master Theorem cannot be used", if this is the case.
\item  $T(n) = 9 T\left(\frac{n}{3}\right) + n^2 \log n $. Use the Master Theorem to find a $\Theta()$ evaluation, or say "Master Theorem cannot be used", if this is the case.
\end{itemize}

\bigskip
\textbf{Answers:}

\begin{enumerate}
\item For $T(n) = 3 T\left(\frac{n}{4}\right) + 3$:
\[
a=3,\quad b=4,\quad f(n)=3.
\]
We compute 
\[
n^{\log_4 3}.
\]
Since $f(n)=3=\Theta(1)$, and $\Theta(1)=O\Bigl(n^{\log_4 3-\epsilon}\Bigr)$ for any $\epsilon>0$, we are in Case 1 of the Master Theorem. So,
{\boldmath\[
\quad T(n)=\Theta\Bigl(n^{\log_4 3}\Bigr)
\]}

\item For $T(n) = 2 T\left(\frac{n}{2}\right) + 3n$:
\[
a=2,\quad b=2,\quad f(n)=3n.
\]
We compute 
\[
n^{\log_2 2}=n.
\]
Since $f(n)=\Theta(n)$, which is the same as $n^{\log_2 2}$, we are in \textbf{Case 2} of the Master Theorem. So,
{\boldmath\[
T(n)=\Theta(n\log n).
\]}

\item For $T(n) = 9 T\left(\frac{n}{3}\right) + n^2 \log n$:
\[
a=9,\quad b=3,\quad f(n)=n^2\log n.
\]
We compute 
\[
n^{\log_3 9}=n^{\log_3 (3^2)}=n^2.
\]
Comparing \( f(n) \) with \( n^{\log_b a} \):
\[
\frac{f(n)}{n^2} = \frac{n^2 \log n}{n^2} = \log n.
\]
Since the extra factor is only \( \log n \) (which is not a polynomial factor), the difference between \( f(n) \) and \( n^2 \) is less than a polynomial factor.

The standard Master Theorem requires that \( f(n) \) be either polynomially smaller or larger than \( n^{\log_b a} \) (i.e., differing by a factor of \( n^\epsilon \) for some \( \epsilon > 0 \)). Since \( \log n \) grows slower than any positive power of \( n \), \textbf{the standard Master Theorem cannot be used.}

\end{enumerate}
\bigskip

\textbf{Exercise 2.}
\begin{itemize}
    \item \(T(n) = 2T(n-1) + 1\), with \(T(0)=1\). Use the iteration method to find a \(\Theta()\) evaluation for \(T(n)\).
    \item \(T(n) = T(n-1) + 1\), with \(T(0)=1\). Use the iteration method to find a \(\Theta()\) evaluation for \(T(n)\).
    \item Give a \(\Theta(\cdot)\) evaluation for the runtime of the following code:
    \begin{verbatim}
i = n
while(i >= 1) {
    for (j = 1; j <= n; j++)
        x = x + 1
    i = i/2
}
    \end{verbatim}
    \item Give a \(\Theta(\cdot)\) evaluation for the runtime of the following code:
    \begin{verbatim}
i = n
while(i >= 1) {
    for (j = 1; j <= i; j++)
        x = x + 1
    i = i/2
}
    \end{verbatim}
\end{itemize}

\newpage
\textbf{Answers:}

\begin{enumerate}
    \item For \(T(n) = 2T(n-1) + 1\), with \(T(0)=1\):
    \[
    \begin{aligned}
    T(n) &= 2T(n-1) + 1\\[1mm]
         &= 2\Bigl[2T(n-2) + 1\Bigr] + 1 = 2^2 T(n-2) + 2 + 1\\[1mm]
         &= 2^3 T(n-3) + 2^2 + 2 + 1\\[1mm]
         &\;\,\vdots\\[1mm]
         &= 2^nT(0) + \sum_{i=0}^{n-1} 2^i\\[1mm]
         &= 2^n + \left(2^n - 1\right)\\[1mm]
         &= 2^{n+1} - 1.
    \end{aligned}
    \]
    So, 
    {\boldmath\[
    T(n) = \Theta(2^n).
    \]}
    
    \item For \(T(n) = T(n-1) + 1\), with \(T(0)=1\):
    \[
    \begin{aligned}
    T(n) &= T(n-1) + 1\\[1mm]
         &= T(n-2) + 1 + 1\\[1mm]
         &\;\,\vdots\\[1mm]
         &= T(0) + n\\[1mm]
         &= 1 + n.
    \end{aligned}
    \]
    So, 
    {\boldmath\[
    T(n) = \Theta(n).
    \]}
    
    \item For the following code:
    \begin{verbatim}
        i = n
        while(i >= 1) {
            for (j = 1; j <= n; j++)
                x = x + 1
            i = i/2
        }
    \end{verbatim}
    the outer loop runs \(\Theta(\log n)\) times (since \(i\) is halved each time) and the inner loop runs \(\Theta(n)\) times per iteration. So, the total runtime is:
    {\boldmath\[
    \Theta(n \log n).
    \]}
    
    \item For the following code:
    \begin{verbatim}
        i = n
        while(i >= 1) {
            for (j = 1; j <= i; j++)
                x = x + 1
            i = i/2
        }
    \end{verbatim}
the outer loop iterates \(\Theta(\log n)\) times with \(i = \frac{n}{2^k}\) in the \(k\)th iteration, and the inner loop executes \(\Theta\left(\frac{n}{2^k}\right)\) operations. So, the total work is:
\[
\sum_{k=0}^{\lfloor \log_2 n \rfloor} \frac{n}{2^k} = n + \frac{n}{2} + \frac{n}{4} + \cdots,
\]
a geometric series with first term \(a=n\) and common ratio \(r=\frac{1}{2}\) that sums to \(\frac{n}{1-\frac{1}{2}} = 2n\), so:
    {\boldmath\[
    T(n) = \Theta(n).
    \]}
\end{enumerate}
\newpage
\textbf{Programming Task 1.}: The input is an array $a_1, a_2, \ldots, a_n$ of numbers.  A  \emph{UP-pair} is a pair $(a_i, a_j)$  so that $1 \le i < j \le n$ and $a_i < a_j$. The task is to count the number of UP-pairs in the array. The ideal solution is to use a modified merge-sort algorithm that counts the number of UP-pairs as it merges the elements. The code first calls a wrapper method, countUpPairs(\textit{A}), which calculates the initial values of \textit{p} and \textit{r} for array \textit{A[]}. The code then recursively calls upon the divideAndCount(\textit{A, p, r}) method, first calculating the index of the midpoint \textit{q} as $p + \frac{r-p}{2}$. The method recursively calls upon the left and right half to split them until $p>=r$, or until there is only 1 element in the given segment of the array (No UP-pairs in an array with length 1 or 0). After the array has been split down to its smallest segments, divideAndCount() calls upon mergeAndCount(A, p, q, r) to count the number of UP-pairs in a given segment. It works the same as the standard merge sort algorithm the main difference here is that while merging, if $L[i] < R[j]$ count is incremented by $nR - j$ where \textit{nR} is the length of the right array segment and \textit{j} is the current index of the right array segment. This works because in a given sorted array, if $L[i] < R[j]$, in otherwords \textit{L[i]} forms a UP-pair with \textit{R[j]}, then all other elements of \textit{R[j]} also form a UP-pair with \textit{L[i]} because \textit{R[j]} is sorted and all elements of R come after the element \textit{L[i]}. This way the algorithm runs in $O(n\log n)$ time as opposed to the approach using nested loops which runs in $O(n^2)$.

\begin{center}
\begin{tabular}{|p{10em}|p{30em}|} 
\hline
\textbf{Input} & \textbf{Output} \\ 
\hline
7, 3, 8, 1, 5 & 4 UP-pairs \newline [1, 3, 5, 7, 8] \\ 
\hline 
input-3.4.txt & 248339 UP-pairs \newline [0, 1, 3, 4, 4, 5, 8, 9, \ldots, 984, 987, 990, 995, 999] \\ 
\hline 
input-3.5.txt & 24787869 UP-pairs \newline [2, 3, 3, 4, 4, 6, 8, 9, \ldots, 9996, 9997, 9998, 9999, 9999] \\ 
\hline
\end{tabular}
\end{center}

\pagebreak
\textbf{Raw Code for Programming Task 1}

\lstset{
    basicstyle=\ttfamily\footnotesize,
    breaklines=true,  % Enables automatic line breaking
    frame=single,     % Adds a border around the code
    tabsize=4,        % Sets tab width
    showstringspaces=false % Removes visible spaces in strings
}

\begin{lstlisting}[language=Java]
import java.util.*;
import java.io.*;

public class Asgmnt3Task1 {
    public static void main(String[] args) throws FileNotFoundException  {
        int[] arr = {7, 3, 8, 1, 5};
        System.out.println("Data Set 1:");
        // countUpPairs sorts the array and returns the number of UP-pairs.
        System.out.println("UP-Pairs count: " + countUpPairs(arr));
        System.out.println("Sorted array: " + Arrays.toString(arr));
        System.out.println();

        // Prompt the user to enter filenames for additional data sets
        Scanner console = new Scanner(System.in);
        System.out.println("Enter filenames (separated by spaces):");
        String inputLine = console.nextLine().trim();
        if (!inputLine.isEmpty()) {
            // Split the input string into individual filenames
            String[] filenames = inputLine.split("\\s+");
            for (String filename : filenames) {
                try {
                    Scanner fileScanner = new Scanner(new File(filename));
                    // The first integer in the file is the number of elements
                    int n = fileScanner.nextInt();
                    int[] fileInputArr = new int[n];
                    // Read the next n integers from the file into the array
                    for (int i = 0; i < n; i++) {
                        fileInputArr[i] = fileScanner.nextInt();
                    }
                    fileScanner.close();
                    // Process the array: sort it and count the UP-pairs using the modified merge sort
                    int count = countUpPairs(fileInputArr);
                    System.out.println("File: " + filename);
                    System.out.println("UP-Pairs count: " + count);
                    System.out.println("Sorted array: " + Arrays.toString(fileInputArr));
                    System.out.println();
                } catch (FileNotFoundException e) {
                    System.err.println("File not found: " + filename);
                }
            }
        }
    }

    /**
     * countUpPairs:
     * Initiates the modified merge sort that counts UP-pairs.
     *
     * @param A the input array of numbers.
     * @return the total number of UP-pairs in the array.
     */
    public static int countUpPairs(int[] A) {
        return divideAndCount(A, 0, A.length - 1);
    }

    /**
     * divideAndCount:
     * Recursively divides the array into halves, counts the UP-pairs in
     * each half, and then counts the UP-pairs that cross the two halves during the merge.
     *
     * @param A the array to process.
     * @param p the starting index.
     * @param r the ending index.
     * @return the number of UP-pairs in the subarray A[p..r].
     */
    public static int divideAndCount(int[] A, int p, int r) {
        if (p >= r)
            return 0;
        int q = p + ((r - p) / 2);
        // Recursively count UP-pairs in the left half, right half, and across the halves.
        return divideAndCount(A, p, q)
                + divideAndCount(A, q + 1, r)
                + mergeAndCount(A, p, q, r);
    }

    /**
     * mergeAndCount:
     * Merges two sorted subarrays A[p...q] and A[q+1...r] while counting UP-pairs.
     * If an element in the left subarray (L) is less than an element in the right subarray (R),
     * then all remaining elements in R will form UP-pairs with that element from L.
     *
     * @param A the array containing the two subarrays.
     * @param p the starting index of the first subarray.
     * @param q the ending index of the first subarray.
     * @param r the ending index of the second subarray.
     * @return the number of UP-pairs counted during the merge.
     */
    public static int mergeAndCount(int[] A, int p, int q, int r) {
        int nL = q - p + 1; // Length of left subarray
        int nR = r - q;     // Length of right subarray

        int[] L = new int[nL];
        int[] R = new int[nR];

        // Copy A[p...q] into L and A[q+1...r] into R
        for (int i = 0; i < nL; i++) {
            L[i] = A[p + i];
        }
        for (int j = 0; j < nR; j++) {
            R[j] = A[q + j + 1];
        }
        int i = 0, j = 0, k = p, count = 0;
        // Merge the two sorted arrays back into A while counting UP-pairs.
        // When L[i] < R[j], all elements after R[j] form an UP-pair with L[i].
        while (i < nL && j < nR) {
            if (L[i] < R[j]) {
                count += (nR - j);  // All remaining elements in R are greater than L[i]
                A[k++] = L[i++];
            } else {
                A[k++] = R[j++];
            }
        }
        // Copy any remaining elements of L into A.
        while (i < nL) {
            A[k] = L[i];
            i++;
            k++;
        }
        // Copy any remaining elements of R into A.
        while (j < nR) {
            A[k] = R[j];
            j++;
            k++;
        }
        // Return the count of UP-pairs found during the merge.
        return count;
    }
}
\end{lstlisting}
\pagebreak

\textbf{Programming Task 2.}


The input is a 2-dimensional  matrix cost[][] of integer numbers, with $n$ rows labeled $0,1, 2, \ldots, n-1$ and $m$ columns labeled $0,1,2,\ldots, m-1$, and also a target cell $(i,j)$. The task is to calculate the minimum cost path to reach cell (i, j) from cell (0, 0). Each cell of the matrix represents a cost to traverse through that cell. The total cost of a path to reach (i,j) is the sum of all the costs on that path (including both source and destination).
We can only traverse down, right,  and diagonally lower  from a given cell. Formally, from a given cell (i, j) one can move to one of the following cells: 

(i+1, j) (down)


(i, j+1) (right)



 and the diagonal moves are to (i+1, j+1), or to  (i+1,j-1).

\medskip

Example:  the input is the matrix cost below and the target cell $(2,1)$.
\medskip


\begin{tabular}{|c|c|c|}
\hline
1  & 2 & 3 \\
\hline
4 & 8 & 1 \\
\hline
1 & 5 & 3 \\
\hline

\end{tabular}
\medskip

Then the minimum cost to go from cell (0,0) to the target cell (2,1) is 9, corresponding to the path (0,0) -- (0,1) -- (1,2) -- (2,1).


Design a dynamic programming algorithm that solves this problem. The input should have the following format

line 1 consists of 4 numbers: n, m, i and j.

This is followed by n rows each one containing m numbers, giving the cost table.

For instance for the above problem the input is

3,3,2,1

1,2,3

4,8,1

1,5,3


Test your program and report in a table the results for the following data sets.

Data set 1:  The above example

Data set 2:  The  numbers from the file input-3.6, available on Blackboard. 

Data set 3:  The  numbers from the file input-3.7, available on Blackboard. 

\pagebreak

\begin{center}
\begin{tabular}{|p{30em}|}
\hline
\texttt{'Assignment3\_T2'}\\[1mm]
Integer Matrix Size : [3][3]\\[1mm]
[ 1, 2, 3 ]\\[1mm]
[ 4, 8, 1 ]\\[1mm]
[ 1, 5, 3 ]\\[1mm]
\textbf{Min cost to cell (2, 1) : 9}\\[2mm]
\hline
Input File Name : input-3.6.txt\\[1mm]
Integer Matrix Size : [6][6]\\[1mm]
[ 3, 1, 1, 1, 1, 1 ]\\[1mm]
[ 1, 4, 2, 3, 5, 1 ]\\[1mm]
[ 9, 1, 2, 3, 4, 5 ]\\[1mm]
[ 1, 7, 2, 5, 4, 4 ]\\[1mm]
[ 1, 1, 1, 1, 1, 1 ]\\[1mm]
[ 1, 7, 1, 7, 1, 7 ]\\[1mm]
\textbf{Min cost to cell (5, 5) : 16}\\[2mm]
\hline
Input File Name : input-3.7.txt\\[1mm]
Integer Matrix Size : [8][5]\\[1mm]
[ 1, 2, 3, 4, 5 ]\\[1mm]
[ 5, 4, 3, 2, 1 ]\\[1mm]
[ 1, 2, 3, 4, 5 ]\\[1mm]
[ 5, 4, 3, 2, 1 ]\\[1mm]
[ 1, 2, 3, 4, 5 ]\\[1mm]
[ 5, 4, 3, 2, 1 ]\\[1mm]
[ 1, 2, 3, 4, 5 ]\\[1mm]
[ 5, 4, 3, 2, 1 ]\\[1mm]
\textbf{Min cost to cell (7, 4) : 20}\\[2mm]
\hline
\end{tabular}
\end{center}

\pagebreak
\textbf{Task 2}
\begin{lstlisting}[language=Java]
import java.io.File;
import java.io.FileNotFoundException;
import java.util.Scanner;

public class Assignment3_T2 {
    public static int rowF = 0;
    public static int colF = 0;

    public static void main(String[] args) throws FileNotFoundException {
        int[][] temp = new int[][]{ {1, 2, 3}, {4, 8, 1}, {1, 5, 3}};
        rowF = 2;
        colF = 1;
        print(temp, temp.length, temp[0].length);
        System.out.println("\nOptimize path (0, 0) to (" + rowF + ", " + colF + ")\n");
        System.out.println("\nMin cost to cell  (" + rowF + ", " + colF + ") : " + costOpti(temp)+"\n");
        
        
        int[][] x = fileInts();
        print(x, x.length, x[0].length);
        System.out.println("\nOptimize path (0, 0) to (" + rowF + ", " + colF + ")\n");
        System.out.println("\nMin cost to cell  (" + rowF + ", " + colF + ") : " + costOpti(x)+"\n");

        x = fileInts();
        print(x, x.length, x[0].length);
        System.out.println("\nOptimize path (0, 0) to (" + rowF + ", " + colF + ")\n");
        System.out.println("\nMin cost to cell  (" + rowF + ", " + colF + ") : " + costOpti(x)+"\n");

    }

    public static int costOpti(int[][] x) {

        int minInc[][] = new int[x.length][x[0].length];

        // Fill edge row with right only movements//
        minInc[0][0] = x[0][0];
        for (int i = 1; i < x[0].length; i++) {// Case only right moves: Top//
            int rightMoves = minInc[0][i - 1] + x[0][i];
            minInc[0][i] = rightMoves;
        }

        // Process array by layer as each element in a row > 0 has group of
        // Moves to choosen: cell0:(2), cell1:(4), cell2:(4), cell3:(4), ..., Last
        // index:(3)

        for (int k = 1; k < x.length; k++) {// all other rows//

            for (int j = 0; j < x[k].length; j++) {// in line//

                if (j == 0) {// First index of row//
                    int downMove = x[k][j] + minInc[k - 1][j];
                    int diag = x[k][j] + minInc[k - 1][j + 1];
                    minInc[k][j] = Math.min(downMove, diag);
                }

                else if (j < x[k].length - 1) {// Not at last index of row//
                    int downMove = x[k][j] + minInc[k - 1][j];
                    int diagUpRight = x[k][j] + minInc[k - 1][j + 1];
                    int diagUpLeft = x[k][j] + minInc[k - 1][j - 1];
                    int leftMoves = x[k][j] + minInc[k][j - 1];

                    minInc[k][j] = Math.min(Math.min(diagUpRight, diagUpLeft), Math.min(downMove, leftMoves));
                }

                else {// Last index of row//
                    int downMove = x[k][j] + minInc[k - 1][j];
                    int diagUpLeft = x[k][j] + minInc[k - 1][j - 1];
                    int leftMoves = x[k][j] + minInc[k][j - 1];

                    minInc[k][j] = Math.min(diagUpLeft, Math.min(downMove, leftMoves));

                }

            }
        }
        // Actually reviews all moves for each square
        // Does full array incase of negative cost areas...

        System.out.println("Min cost path summations processed...");
        print(minInc, minInc.length, minInc[0].length);

        return minInc[rowF][colF];
    }

    // Done : builds array from file correctly//
    public static int[][] fileInts() throws FileNotFoundException {
        Scanner scnr = new Scanner(System.in);

        System.out.print("Input File Name : ");
        String userVar = scnr.next();

        try {
            while (!new File(userVar).exists()) {
                System.out.print("Input File Name : ");
                userVar = scnr.next();
            }
            System.out.println();
            Scanner scnrX = new Scanner(new File(userVar));

            // Grab size row x col first//
            int n = scnrX.nextInt();
            int m = scnrX.nextInt();
            // Public static goal index//
            rowF = scnrX.nextInt();
            colF = scnrX.nextInt();

            int x[][] = new int[n][m];

            for (n = 0; n < x.length; n++) {
                for (m = 0; m < x[0].length; m++) {
                    if (scnrX.hasNextInt()) {
                        x[n][m] = scnrX.nextInt();
                    } else {// Out Of Room//
                        n = x.length;
                        m = x[0].length;
                    }
                }
            }

            return x;
        } catch (FileNotFoundException e) {
            System.out.println("Error File Not Found...");
            return null;
        }
    }

    // Print 2D int array//
    public static void print(int[][] y, int n, int m) {
        System.out.println("Integer Matrix Size : [" + n + "]" + "[" + m + "]");

        int i;
        int j;

        for (i = 0; i < n; i++) {
            System.out.print("[ ");

            for (j = 0; j < m; j++) {

                if (j != 0) {
                    System.out.print(", ");
                }
                System.out.print(y[i][j]);

            }

            System.out.println(" ]");

        }

    }
}
\end{lstlisting}

\end{document}
